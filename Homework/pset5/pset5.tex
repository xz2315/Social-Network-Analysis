\documentclass[11pt]{article}
\usepackage{amsmath,amssymb,amsthm}
\usepackage[utf8]{inputenc}
\usepackage{graphicx}
\usepackage{xcolor}

\DeclareMathOperator*{\E}{\mathbb{E}}
\let\Pr\relax
\DeclareMathOperator*{\Pr}{\mathbb{P}}

\newcommand{\eps}{\varepsilon}
\newcommand{\inprod}[1]{\left\langle #1 \right\rangle}
\newcommand{\R}{\mathbb{R}}


\newcommand{\handout}[5]{
  \noindent
  \begin{center}
  \framebox{
    \vbox{
      \hbox to 6.38in { \textcolor{black}{CS 134: Networks } \hfill #4  }
      \vspace{1mm}
      \hbox to 6.42in { #3 {\hfill #2 } }
      \vspace{0.0mm}
      \hbox to 6.38in { { \hfill} }
    }
  }
  \end{center}
  \vspace*{0mm}
}

\newcommand{\pset}[3]{\handout{{#1}}{\textcolor{red}{Due: #2}}{\textcolor{black}{#3}}{\textcolor{gray}{\textbf{Problem set #1}}}}

\newtheorem{theorem}{Theorem}
\newtheorem{corollary}[theorem]{Corollary}
\newtheorem{lemma}[theorem]{Lemma}
\newtheorem{observation}[theorem]{Observation}
\newtheorem{proposition}[theorem]{Proposition}
\newtheorem{definition}[theorem]{Definition}
\newtheorem{claim}[theorem]{Claim}
\newtheorem{fact}[theorem]{Fact}
\newtheorem{assumption}[theorem]{Assumption}
\newtheorem{remark}[theorem]{Remark}

\newtheorem*{theorem*}{Theorem}
\newtheorem*{notation*}{Notation}
\newtheorem*{assumption*}{Assumption}
\newtheorem*{fact*}{Fact}
\newtheorem*{claim*}{Claim}
\newtheorem*{definition*}{Definition}
\newtheorem*{exercise*}{Exercise}
\newtheorem*{lemma*}{Lemma}
\newtheorem*{remark*}{Remark}

% 1-inch margins, from fullpage.sty by H.Partl, Version 2, Dec. 15, 1988.
\topmargin 0pt
\advance \topmargin by -\headheight
\advance \topmargin by -\headsep
\textheight 8.9in
\oddsidemargin 0pt
\evensidemargin \oddsidemargin
\marginparwidth 0.5in
\textwidth 6.5in

\parindent 0in
\parskip 1.5ex


\usepackage[margin=.9in]{geometry}
\usepackage{amsmath}
\usepackage{tikz}
\usepackage{float}
\usepackage{filecontents}
\usepackage{pgfplots, pgfplotstable}
%\usepgfplotslibrary{statistics}
\usepackage[T1]{fontenc}
\usetikzlibrary{calc,intersections,through,backgrounds,shadings}
\usetikzlibrary{shapes.geometric}
\usetikzlibrary{through}

\usepackage[nofiglist,nomarkers]{endfloat}
\renewcommand{\efloatseparator}{\mbox{}}

\usepackage{exercise}
\renewcommand{\ExerciseHeader}{{ \textbf{
\ExerciseName \ \ExerciseHeaderNB \ExerciseHeaderTitle
\ExerciseHeaderOrigin.}}}

\usepackage{pgfplots}
\pgfplotsset{
%  compatgraph=newest,
  xlabel near ticks,
  ylabel near ticks
}



\begin{document}

\pset{5}{\textbf{Wednesday 3/8/2017}}{{Prof.\ Yaron Singer}}


\paragraph{1.} \textbf{(20 points)}  Use the Iterated Deletion of Strictly Dominated Strategies to find all Nash equilibria of the following game:
$$\begin{array} {|c|c|c|c|} \hline
& \text{L} & \text{M} & \text {R}\\ \hline
\text{U} & 2,4 & 2,1 & 3,2 \\ \hline
\text{D} & 1,2 & 3,3 & 2,4 \\ \hline
\end{array}$$
Qualitatively describe why this algorithm never eliminates a strategy that is part of a Nash equilibrium.

\paragraph{2.} \textbf{(25 points)} Consider five strangers traveling in the last row of a bus from Boston to New York City. Each of them has the option to read a book or to sleep during the trip. Each of them controls a bulb that provides $60$ watts of light, while the bulb of any immediate neighbor provides $30$ watts. Each of them prefers as much light as possible if more than 100 watts of light reaches them (they can't read with less than 100 watts of light, but as long as they can read, they want to read with as much light as possible) but as little as possible otherwise (if they can't read, they want to sleep with as little light as possible).

\begin{itemize}
\item[\textbf{a.}] \textbf{(7 points)}  Transform this situation into a game (that is, list the set of players, the set of strategies and an appropriate payoff function).
\item[\textbf{b.}] \textbf{(6 points)}  List any/all dominated strategies you can find in this game. 
\item[\textbf{c.}] \textbf{(6 points)}  List any/all  dominant strategies you can find in this game.
\item[\textbf{d.}] \textbf{(6 points)}   List any/all equilibria you can find in this game.
\end{itemize}
 
\paragraph{3.} \textbf{(25 points)} Redo Exercise 2 in entirety assuming that the five strangers are seated in a round table (so that each passenger has both a right and left neighbor), instead of in a row.


\paragraph{4. Borrow-a-Book Network Game} \textbf{(30 points)} A \emph{network game} $(N, g)$ is a special class of game in which 

\begin{itemize}
\item the set of players $N$ is connected by a network $g$,
\item the set of available strategies for each player is $\{0,1\}$, and
\item the payoff function $u_i(x_i, x_{N_i(g)})$ of each player $i$ depends only on the action $x_i$ of player $i$ and the actions\footnote{$x_{N_i(g)}$ is the vector that contains all elements of $x$ that correspond to the neighbors of $i$} $x_{N_i(g)}$ of her neighbors $N_i(g)$ in the network $(N,g)$.\end{itemize}

In other words, the utility of each node is defined on both its own action and the actions of its neighbors.

Each player in a \emph{borrow-a-book network game} chooses whether to buy a book. If a player does not buy the book, then he can freely borrow it  from any of his neighbors who bought it. Indirect borrowing is not permitted, so the player cannot borrow a book from a neighbor of a neighbor. If none of a player's neighbors has bought the book, then the player would prefer to pay the cost of buying the book himself rather than not having access to the book. This is problem is a classic \emph{free rider} problem, but defined on a network. Formally:

 $$\begin{array} {l l }
u_i(1, x_{N_{i(g)}})=1-c &  \\
u_i(0, x_{N_{i(g)}})=\begin{cases}\begin{array} {l l} 1 & \text{if} \ x_j=1 \ \text{for some} \ j\in N_i(g) \\  0 & \text{if} \ x_j=0 \ \text{for all} \ j\in N_i(g)\\ \end{array}\end{cases}
\end{array}$$
 
where $1>c>0$ denotes the cost of buying the book. 

\begin{itemize}
\item[\textbf{a.}] \textbf{(5 points)} Characterize the set of Nash Equilibria of a borrow-a-book game in which the players are modeled as and linked in the form of a clique. 
\item[\textbf{b.}] \textbf{(5 points)}  Characterize the set of Nash Equilibria of a borrow-a-book game in which the players are modeled as and linked in the form of a ring. 
\item[\textbf{c.}]  \textbf{(10 points)} For any network $G$, let $m(G)$ be the minimum number of books bought in any equilibrium of the borrow-a-book game in which players are linked by $G$. Suppose we obtain $G'$ by adding some links to $G$. Is it possible that $m(G')>m(G)$? If not, (qualitatively) prove why not. If so, give an example.
\item[\textbf{d.}]  \textbf{(10 points)} Consider a borrow-a-book game whose players are linked by the graph $G=(N,g)$. Give a condition (or set of conditions) on the subset $N'$ of $N$ ($N'\subset N$) that precisely guarantees that the set of players $N'$ are the only ones buying the book in some equilibrium. Be as precise as possible.
\end{itemize}

\end{document}

